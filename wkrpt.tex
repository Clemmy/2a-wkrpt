\documentclass[12pt]{article}
\usepackage{wkrpt}
\begin{document}


\title{Backing Database System for a Messaging Platform Startup}
{
	Ten Thousand Coffees\\
	Toronto, ON
}
{
	Clement Hoang\\
	2B Software Engineering\\
	20531116\\
	c8hoang\\
	May 4th, 2015
}


\letter{Backing Database System for a Messaging Platform Startup}{2A}{Ten Thousand Coffees}{Software Engineering}
{
	\noindent
	Clement Hoang\\
	333 King Street North\\
	Waterloo, ON. 2Z1 N2J
}
{
	Ten Thousand Coffees is startup whose main product is a social networking platform directed towards connecting students with industry professionals. Employed as a member of the core development team for Ten Thousand Coffees, I worked on improving the site and starting experiments over the course of the term. One of the tasks that I was entrusted with was the addition of messaging functionality to Ten Thousand Coffee's online platform.
}
{
	From the the initial planning phase of the messaging system until now, there were several changes of requirements, as well as a limited budget for development. One of the fundamental decisions to make during planning phase was the database system to utilizes for this project. This report analyzes two database systems suitable for backing the messaging platform, and identifies the more appropriate alternative for the project.
}
{
	I would like to thank my co-workers at Ten Thousand Coffees for their continuous guidance during my stay with them. Additionally, I would like to especially thank Elliott Garcea, the lead engineer at Ten Thousand Coffees, for his detailed discussion about our platform's design decisions with me. Finally, I would like to thank my friend, Raymond Wan, for answering a lot of questions that I have about relational database management systems.
}
{
	Clement Hoang, 20531116
}


\tocsection{Executive Summary}
blah blah
The Executive Summary is a one-page summary of the report's highlights, including the purpose and scope of the report, the major points in the report, highlights of the conclusions, and highlights of the recommendations. It is self contained, and is meant to be read and understood by someone (e.g., a company executive officer) who is not likely to read the report itself.

- something about startups
- something about mysql and mongodb (background about traditional relational vs non-relational debate, hard to decide without ocntext, both solve different use cases)
- something about how message is modelled

\newpage


\toc
% \lof
% \lot


\pagenumbering{arabic}
\section{Introduction}
For several decades, the amount of technological startups has been growing and exploring a vast amount of business models to meet the needs of an ever-changing market. From dating sites to smart-watches, there is a business for almost every idea imaginable. Upon closer inspection of the aforementioned startups, there are noticeably many web applications which are primarily focused on messaging, or, at the very least, include a messaging component. To enable messaging capabilities in a web application, there needs to be a web-server to act as the communication bridge between different users, a database to persist user and message data onto for later usage, and a user interface to allow different people to interact with the web application and message other users. The database, in particular, is very important because it serves as a central store for writing and retrieving user generated content such as message content, chat history logs, and various other datasets for use by the application; without the database, a messaging application simply ceases to function.

However, this is where the startup aspect of Ten Thousand Coffees comes into play. Ten Thousand Coffees is a web application that enables users on the site to connect with one another and invite other parties for coffee chats. As a growing company striving to survive in a fierce environment, the importance of a flexible, scalable, and affordable solution is paramount. This is because startups are limited in budget, and have the need the for high growth. In order to have be successful, a startup must be efficient in using the resources that it can afford, and make decisions that enable them to scale both quickly, and affordably, so that they could outperform the competition. For the Ten Thousand Coffees platform, many technical decisions were made during the planning phase, and one such decision was to use MongoDB, a new, open-sourced NOSQL database that is based on a document object model that features horizontal scalability as well as a dynamic schema that allows for agile development. MongoDB was chosen over a traditional SQL database such as MySQL. In contrast, SQL databases have been around for much longer than NOSQL databases, and therefore have gained a lot of enterprise users. A SQL database such as MySQL and a NOSQL database such as MongoDB have completely different design principles, and this report will explore why Ten Thousand Coffees chose MongoDB over MySQL.

\section{Background}
As mentioned in the introduction earlier, Ten Thousand Coffees is a social platform whose mission is to connect the youth of today with the leaders of tomorrow over coffee chats. The solution that the company decided on to approach this problem was a web platform that allows the discovery of possible coffee candidates, with a messaging and scheduling system that allows users to first chat online, before setting up a meeting via the platform and meeting up in real life. The team consists of a business team and a development team, of which contain five and seven employees correspondingly. As a startup with a small team and therefore, limited manpower, there were certain requirements that the chosen database would have to meet in order to be feasible for the platform. These requirements include the ability to prototype rapidly, to be able to scale well, and to be easy-to-use, while still meeting the performance standards for a modern web application. In order to meet these requirements, the team unanimously decided on the MEAN stack, which is a fullstack Javascript framework that became popular over the past few years. For more information on how a web application architectured with the MEAN stack works, see Appendix A.

\section{Rapid Prototyping}
The ability to prototype rapidly is critical to a startup because at the early stages of a company, there is no absolute path from beginning to end goal. This means that the requirements will continuously change and adapt based on the feedback from consumers as well as data gathered by analytics. Because of the need to be able to iterate and improve the platform quickly based on changing requirements, the database management system that is used will need to be flexible in that data can quickly and easily be migrated to support the changing requirements. In this section, the abilities of MongoDB and MySQL to meet the requirements of fast prototyping will be explored.

\subsection{Schema Flexibility}
The criterion of \'schema flexibility\' refers to the ability of a database to adapt to the changing of the data schema. This is an important measure when comparing the benefits of each database system for a messaging platform startup because through the lifetime of the application, the requirements will continuously evolve to meet the demands of the consumers, and it is also not uncommon for bugs at the schema design level to be introduced where a database schema change is necessary in order to fix it.

For example, an early prototype of a messaging platform may only support conversations in which one user can message only one other user. Obviously, most dedicated messaging platforms will eventually improve its functionality to support features such as multi-user messaging, but as an early iteration, to be able to message only one other user is sufficient. Another scenario in which a schema change may be necessary is when a boolean flag field needs to be added into the schema to contain additional state knowledge will helps to solve certain bugs. In these cases, the database management system that is used should be able to migrate or adapt data from the schema of an earlier iteration with a schema of a later iteration with relative ease in order to be a good fit for a startup.

\subsubsection{MongoDB}


\subsubsection{MySQL}

-mention something about history loss? - to migrate without losing history


blah blah see appendix B (for brevity, a lot of details have been ommitted. See appendix B for case study.)

% \subsection{Schema Complexity}
% dsfsdfa
% -show mongo schema vs mysql schema, more tables required  etc.

\section{Scaling}

\section{Performance}
The Main Sections of the report are the heart of the report, containing detailed descriptions of the technical decision being reported. In an analysis report, the Main Sections start by elaborating on the problem identified in the Introduction , expanding on the problem definition and on background material relevant to understanding the problem. They describe the analytical study or the experiment performed, and explicate and justify any assumptions made. They conclude by reporting the results of the study, drawing conclusions, and making recommendations. To be complete, an analysis report should identify any threats to the validity of the analysis' results, and should describe the costs and consequences of following, and of not following, the recommendations. In a synthesis report, the Main Sections start by elaborating on the problem identified in the Introduction, expanding on the problem definition to include rigorous and precise specifications of the design criteria and constraints. The sections describe your solution to the problem, including an explanation of how your solution satisfies the design criteria and constraints. They also report alternative designs and solutions that were considered, and they explain why the proposed solution is most favourable in the given context. To be complete, a synthesis report should identify any limitations of the solution proposed.



\section{Conclusions}
CONCLUSIONS
blah blah something results (summarize results and constrast)
it is clear that mongodb matches the criteria better
mysql suited for other purposes...such as complexity or something else, but it is not a good fit in an environment that is continuously changing and evolving
\cite{smpl}

\section{Recommendations}
RECOMMENDATIONS

\newpage


\addcontentsline{toc}{section}{\refname}
\bibliography{wkrpt}
\newpage


\tocsection{Acknowledgements}
ACKNOWLEDGEMENTS
\newpage


% \appendix{APPENDIX INDEX}{APPENDIX NAME}
% APPENDICES
% \newpage
\appendix{A}{The MEAN Stack}
will discuss monoglot, etc.

[reference for each of the 4 things](or put this in appendix)

 In order to build the Ten Thousand Coffees web application, the MEAN stack was chosen. the mean stack is a set of techs blah blah

blah blah...describe stack a bit/include some diagrams of mean stack
what problems were specifically met (similar to introduction, but in more detail (i.e. need fast lookup, team of 6 devs, not a lot of dev manpower, etc.))/ requirements

\newpage

\appendix{B}{Case Study of Schema Migration at Ten Thousand Coffees}
blah blah

\newpage

\end{document}
